\chapter{Conclusions \& Future Work}\label{ch:conclusion}
Initial results of benchmark testing prove to be very encouraging with
regard to the 







In conclusion, I concluded that this conclusion is too short and off
topic.








[left over from this chapter, but out of place here]
[This paragraph feels like it belongs somewhere else]
There are two main reasons for modeling a library instead of including
it as part of the input program.  The first reason is to abstract away
interaction with the operating system and hardware.  Directly
including the pthread library as part of input programs would require
a model of the OS system calls, kernel data, an environmental behavior
that the pthread library accesses.  The engineering cost of doing this
is greater than model the pthread library itself.  The second reason
is that modeling a library greatly reduces the computational
complexity of verification.  A brief glance at the pthread library
source reveals much longer, more complex code than the model designed
in Chapter~\ref{ch:modeldesign}.

[leftover from intro]
[Remaining is leftover stuff -- may be useful when revising intro]
This is supported by the accuracy seen in the empirical results of
initial implementation testing.  My proposed research will extend
SMACK with a model of a common subset of the pthreads API, enabling
verification of C/C++ programs utilizing the Pthreads library.
Extending SMACK to provide support for programs using pthreads has not
only provided the SMACK community with support for a common
concurrency paradigm, but also presents an opportunity to investigate
general constructs useful for modeling concurrency.  These constructs,
which will necessarily result from extending SMACK, will lend
themselves to modeling other concurrency libraries like OpenMP and
MPI.  Further, the resulting constructs should be general enough for
use within other verification tools. 



%%% Local Variables: 
%%% mode: latex
%%% TeX-master: "thesis"
%%% End: 
