\chapter{Conclusions \& Future Work}\label{ch:conclusion}
Initial results of benchmark testing prove to be very encouraging with
regard to the accuracy of the support for the pthread library
implemented in SMACK.  With a vast majority of benchmark tests
verifying correctly, the new implementation seems ready for real-world
use on programs using the core functionality of the pthread library.
These results decidedly demonstrate that an the pthread library can be
accurately modeled using the primitives available in the SMACK
modeling environment.

Despite this initial success, a large portion of the complete API
remains unmodeled.  This will need to be completed before pthread
support can be considered adequate for industrial use.  With a
suitable framework already in place, the remainder of the  API should
be relatively straightforward to implement.

Indeed, support for concurrency within SMACK still has a long way to
go.  Performance is clearly inferior to the best concurrent program
verifiers, and will require coordination with the Corral team in order
to improve. In addition, integrating the lock set analysis implemented
by Pantazis Deligiannis should significantly improve performance by
eliminating interleavings of locked critical sections.  Initial
testing of the lock set analysis integration has been performed, and
the results are promising.

With this additional work, support for programs utilizing the pthread
library should reach the level of accuracy and performance realized by
the SMACK project as a whole.

%%% Local Variables: 
%%% mode: latex
%%% TeX-master: "thesis"
%%% End: 
