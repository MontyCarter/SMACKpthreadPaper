\chapter{Implementation}\label{ch:implementation}
With the model of the pthread library in place, the next step in my
research was to look forward and consider the actual implementation of
phtread library support as part of the SMACK verifier.  This chapter
provides an overview of the actual implementation of pthread library
support within SMACK.  I begin the chapter with a specification of the
goals targeted for the pthread extension to SMACK.  Next, I describe
the mechanism used for including the model within SMACK.  \iffalse
[This is followed by a discussion of changes required in the SMACK
tool itself to support the new model.]\fi  I conclude the chapter with
a description of the current state of the implementation in relation
to the implementation goals. 

\section{Implementation Goals}\label{sec:implementationgoals}
There were two main motivations encouraging the implementation of
support for programs utilizing pthreads within SMACK.  The first, as
mentioned in Chapter~\ref{ch:introduction}, were requests from the
SMACK user community.  The second was to improve SMACK's performance
in the SVCOMP software verification competition.  SVCOMP is an
international software verification, and is co-located with the TACAS
conference, competition.  Last year SMACK made its debut appearance at
SVCOMP, but had several categories in which it was unable to compete;
pthreads was one of these categories.

The specific needs of these two motivators largely overlap; both
require support for the most commonly utilized functionality of the
pthread library.  To assess the most commonly utilized functionality,
I collected benchmarks from SVCOMP, as well as from the Inspect and
CIVL projects to diversify the sample of benchmarks.  This collection
was then scanned for calls into the pthread API.  A count of the
number of occurrences for each call was generated, and the pthread API
procedures prioritized for modeling according to their frequency of
occurrence within this sample set. 

This analysis yielded some good insight into the commonly utilized
functionality of the pthread library.  As expected, the basic
procedures such as \lstinline[breaklines]|pthread_create()|,
\lstinline[breaklines]|pthread_join()|,
\lstinline[breaklines]|pthread_mutex_lock()| and 
\lstinline[breaklines]|pthread_mutex_unlock()| were at the top of the
list. However, functionality such as error handling options as well as
attribute setters/getters controlling the behavior of things such as
multiprocess mutex scope and thread scheduling had very few calls, if
any.  As a result, these type of features were left as future
work. The final set of modeled API calls can be seen in
Section~\ref{sec:implementationstatus}.  

\section{Model Inclusion}
As mentioned in Section~\ref{sec:modelingenvironment}, libraries
included by input programs are not typically linked when SMACK
generates LLVM bytecode.  The functions declared in these library
headers go undefined, and SMACK models their behavior as returning a
nondeterministic value, though not affecting program state.  In this
regard the pthread library was no different.  SMACK would treat a call
to \lstinline|pthread_create()| as simply returning a nondeterministic
integer, with no threads spawned or functions called. 

Generally, the strategy for adding a model to SMACK is as simple as
providing a definition for the functions declared in library headers.
Once a definition has been provided for a function included in a
library header, SMACK will find the function definition and translate
the function definition with the rest of the input program.  As a
result, a call to such a function will use the modeled behavior,
rather than returning a nondeterministic value.

Because SMACK uses Clang to compile input programs to LLVM bytecode,
these externally defined functions can be referenced by the input
C/C++ program the same way that externally defined functionality is
added to any C/C++ program. Header files are included to declare the
functions, and calls to the functions are resolved by the linker to
point the definitions contained in the compiled library file.

This strategy of providing a replacement library containing a model
of the original library is precisely how the pthread model discussed
in Chapter~\ref{ch:modeldesign} was added to SMACK.  When SMACK calls
Clang, it passes the \emph{-I} switch with the path to the replacement
\lstinline[identifierstyle=\color{black}]|pthread.h|.  This causes the
SMACK version of \lstinline[identifierstyle=\color{black}]|pthread.h|
to be included, rather than searching the default include paths and
using the system \lstinline[identifierstyle=\color{black}]|pthread.h|.
With our declarations of the pthread API included, the compiled input
program bytecode is then linked with the bytecode compiled from the C
source file containing our model of the pthread library. As a result,
the AST seen by SMACK during translation to Boogie contains the
pthread library models described in Chapter~\ref{ch:modeldesign}.

\iffalse
\section{SMACK Modification?}
[Maybe delete this section, depending on time]
[Do a section here on init funcs?]

In addition to linking the pthread model during compilation of LLVM IR
bytecode, there were[are?] two additional problems related to LLVM IR
to Boogie translation [how to say smack itself?] that needed to be
addressed.

The first problem is related to the initialization of the
\lstinline|$pthreadStatus| array, which sets all threads to an
\emph{uninitialized} state, as described in
Section~\ref{sec:executionenvironment}.  This initialization must
be performed from a within a call to a function, as neither C/C++ nor
Boogie allow statements to be executed in the global declaration
scope.  To ensure this initialization happens before any code other
input program code is evaluated, this initialization of
\lstinline|$pthreadStatus| must be injected as the first statement in
\lstinline|main()|.  This required extending SMACK itself with an
additional C level Boogie injection function, similar to those
described in Section~\ref{sec:clevelmodeling}.  The added functionality
causes SMACK to search for function definitions whose function name
starts with \lstinline|__SMACK_init_func_|.  Any such function
definition encountered while parsing the AST causes SMACK to insert a
call to the function as the first line of \lstinline|main()|.  This
allows such functions to be defined in modeled libraries, rather than
having to modify the source of each input program requiring such an
initialization.  It is using this new functionality that
\lstinline|$pthreadStatus| is initialized. 

The second problem is the ability of a function call to inform SMACK
that the thread of execution is stopping, such as with the
\lstinline|exit()| system call, or \lstinline|pthread_exit()|.  As
with modeling the operating system environment's thread ID coordination,
terminating a thread of execution is similarly the operating system's
responsibility.  When the operating system receives the
\lstinline|exit()| system call, it proceeds to swap out the process,
and mark it as \emph{stopped}.  There is no Boogie IVL instruction for
this, so it must be modeled.  There are several proposals for
addressing this, but it currently remains unimplemented.
\fi

\section{Implementation Status}
\label{sec:implementationstatus}
Currently, there are 15 pthread API calls modeled in the pthread
extension to SMACK.  These 15 calls represent 2959 of 3257 of all
pthread API calls found in the SVCOMP, Inspect, and CIVL
benchmarks. At roughly 91\%, this covers a significant majority of the
calls into the pthread API library found in the benchmark programs.
Further, some of the missing 300 were explained with a quick check
through the benchmarks which revealed calls to functions like
``\lstinline|my_pthread_mutex_lock()|''.  Despite this, there remain
several pthread API functions unimplemented, such as those alluded to
in Section~\ref{sec:implementationgoals}. 


In its present form, the pthread extension to SMACK models the
following pthread API calls:

\begin{multicols}{2}
\begin{itemize}
\item \lstinline|pthread_create()|
\item \lstinline|pthread_self()|
\item \lstinline|pthread_exit()|
\item \lstinline|pthread_join()|
\item \lstinline|pthread_mutex_init()|
\item \lstinline|pthread_mutexattr_init()|
\item \lstinline|pthread_mutexattr_settype()|
\item \lstinline|pthread_mutex_lock()|
\item \lstinline|pthread_mutex_unlock()|
\item \lstinline|pthread_mutex_destroy()|
\item \lstinline|pthread_cond_init()|
\item \lstinline|pthread_cond_wait()|
\item \lstinline|pthread_cond_signal()|
\item \lstinline|pthread_cond_broadcast()|
\item \lstinline|pthread_cond_destroy()|
\end{itemize}
\end{multicols}

With the most commonly used pthread functions modeled, the pthread
extension to SMACK is certainly ready for experimental use, and even
real-world use with programs that don't fully utilize the pthread
API. 

%%% Local Variables: 
%%% mode: latex
%%% TeX-master: "thesis"
%%% End: 
