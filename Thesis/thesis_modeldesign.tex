\chapter{Model Design}\label{ch:modeldesign}
[Good]
In this chapter, I discuss my research efforts toward developing a
model of the pthread library in SMACK.  I begin by introducing details
of the modeling framework used to create the pthread library model.
Having introduced this, I proceed to discuss the process of modeling
concurrency in the pthread library over the lifecycle of a pthread.

\section{Modeling Framework}
[Good]
As described in Chapter~\ref{ch:background}, SMACK consumes LLVM IR
bytecode, builds a model of the input program, and generates a Boogie
IVL model of the input program.  The design of this toolchain leaves
us with several levels at which modeling semantics can be introduced.
At the C level, external library functions that are otherwise
undefined can be given a definition, rather than verifying the whole
original source library. At the SMACK translation level, undefined
functions can be intercepted so SMACK can perform predefined
alterations to the translated Boogie program rather directly being
translated from the LLVM IR representation.  At the translated Boogie
level, additional variables and instructions can be added to model
environmental functionality that is not defined at the C level.
Finally, there can be additional functionality modeled by the Boogie
IVL solver itself. 

To provide some context for the discussion of the modeling process,
I'll briefly introduce Boogie IVL, and then discuss the mechanisms
used for modeling, such as the Corral concurrency primitives and C
level Boogie injection.

\subsection{Boogie IVL}
[Good]
Perhaps the most illustrative way to introduce Boogie IVL may be to
show an input C program and the relevant portions of the Boogie output
that SMACK returns.
\begin{figure}[h]
\centering
\caption{SMACK Translation of C Program}\label{fig:cToBoogie}
\begin{subfigure}[b]{.45\textwidth}
\centering
\caption{Input C Code}\label{fig:cToBoogie_a}
\begin{lstlisting}


void main() {
  int *x, *y;
  x = malloc(sizeof(int));
  y = malloc(sizeof(int));
  *x = 1;
  *y = 2;
  assert(*x == 1);
}
\end{lstlisting}
\end{subfigure}
~
\begin{subfigure}[b]{.45\textwidth}
\centering
\caption{Boogie Code from SMACK}\label{fig:cToBoogie_b}
\begin{lstlisting}[language=boogie]
var $M: [int]int;

procedure main() {
  var $x, $y: int;
  call $x := $malloc(4);
  call $y := $malloc(4);
  $M[$x] := 1;
  $M[$y] := 2;
  assert($M[$x] == 1);
}
\end{lstlisting}
\end{subfigure}
\end{figure}

Figure~\ref{fig:cToBoogie_b} shows a reduced version of the Boogie
code as translated by SMACK from the input source depicted in
Figure~\ref{fig:cToBoogie_a}.  Translating source language programs
into Boogie IVL programs like this provides a common language on which
to perform verification for solvers like Corral, Boogie verifier, and
Duality.  

\subsection{Corral Concurrency Primitives}
[Good]
Corral is a state of the art solver for Boogie IVL programs.  One of
Corral's recent development efforts has been to improve support for
verifying concurrent programs.  To this end, Corral has extended
Boogie IVL, and recognizes five concurrency primitives that are not
part of the base language.  These are [choose whether to list here or
in background]: 

\begin{itemize}
\item \lstinline|async call| \emph{func}\lstinline|(|\emph{...}\lstinline|)| - Asynchronously calls \emph{func} with the parameter list \emph{'...'}
\item \lstinline|corral_atomic_begin()| - Begins an atomic block
\item \lstinline|corral_atomic_end()| - Ends an atomic block
\item \lstinline|corral_getThreadID()| - Returns the ID of the calling thread.
\item \lstinline|corral_getChildThreadID()| - Returns the ID of the
  thread most recently spawned in the calling procedure. 
\end{itemize}

\begin{figure}[h]
\centering
\caption{Corral Concurrency Primitives in Action}
\label{fig:corralprimitives}
\begin{lstlisting}[language=boogie]
var $x: int;

procedure f() 
  modifies $x;
{
  var $tid: int;
  var $tmp: int;
  call $tid := corral_getThreadID();
  call corral_atomic_begin();
  $x := $tid;
  $tmp := $x
  call corral_atomic_end();
  assert($tmp == $tid);
}

procedure main() {
  var $ch1, $ch2: int;
  $x := 0;
  async call f();
  $ch1 := corral_getChildThreadID();
  async call f();
  $ch2 := corral_getChildThreadID();
  assert($x == $ch1 || $x == $ch2);
}
\end{lstlisting}
\end{figure}

Figure~\ref{fig:corralprimitives} demonstrates the usage of each of
these primitives within the context of a complete Boogie program.
This program initializes the global variable \lstinline|$x| as 0, then
makes two asynchronous calls to the function \lstinline|f()| and
records the thread ID of each of the spawned threads of execution.  In
\lstinline|f()|, each of the spawned threads records their thread IDs.
Each thread then starts an atomic block, where it stores its thread ID
to the global \lstinline|$x| and then loads global \lstinline|$x| into
the local \lstinline|$tmp|.  The \lstinline|assert()| in
\lstinline|f()| should always succeed, since \lstinline|$x| is stored
and loaded within an atomic block. 

However, this program contains a bug.  Because there is no barrier
forcing the parent to wait for the child threads to execute, it is
possible that \lstinline|$x| won't be set to the thread ID of either
children by the time \lstinline|assert()| is called in
\lstinline|main|.

It is this additional support for concurrency that warranted selection
of Corral as the solver to target for extending SMACK to support pthreads.

\subsection{C Level Modeling}
[Ok.  Could use another pass or two]
Because there are several layers at which model semantics can be
introduced, SMACK has introduced several special C level functions 
that don't get directly translated into Boogie code, allowing users to
introduce program semantics at the lower levels of the modeling
framework by writing code at the C level.  This allows environment and
library models to be completely written as C input programs, which
avoids modifying SMACK's source code directly to implement such
models.

An example of this is \lstinline|malloc|.  Rather than having SMACK
itself append a memory model and \lstinline|malloc| definition to each
translated Boogie program, these semantics are instead defined in a
header file that gets included from each input program.  Clang
compiles this included header file, so when SMACK begins translation,
the memory model and \lstinline|malloc| are present in the AST,
leaving SMACK with only the responsibility of translating the LLVM IR.

There are several C level functions that SMACK treats specially when
seen in the LLVM IR AST.  Upon seeing these, SMACK performs special
transformations to the Boogie translation, rather than translating the
function calls as is.  Two of these functions were particularly
important for the pthread extension to SMACK. 


\begin{itemize}
\item \lstinline|__SMACK_code(char* format, ...)| - Injects the string
  \lstinline|format| in the Boogie code.
\item \lstinline|__SMACK_top_decl(char* format, ...)| - Inserts the
  string \lstinline|format| as a global declaration
\end{itemize}

\begin{figure}[h]
\centering
\caption{Injecting Boogie Code from C}\label{fig:cinjToBoogie}
\begin{subfigure}[b]{1\textwidth}
\centering
\caption{Boogie Injecting C Code}\label{fig:cinjToBoogie_a}
\begin{lstlisting}
void main() {
  __SMACK_top_decl("var $globalVar: int;");
  int *x;
  x = malloc(sizeof(int));
  *x = 1;
  __SMACK_code("$globalVar := @", *x);
  __SMACK_code("@ := $globalVar + 1", *x);
  assert(*x == 2);
}
\end{lstlisting}
\end{subfigure}
~
\begin{subfigure}[b]{1\textwidth}
\centering
\caption{Injected Boogie Translation}\label{fig:cinjToBoogie_b}
\begin{lstlisting}[language=boogie]
var $globalVar: int;
var $M: [int]int;

procedure main() {
  var $x: int;
  call $x := $malloc(4);
  $M[$x] := 1;
  $globalVar := $M[$x];
  $M[$x] := $globalVar + 1;
  assert($M[$x] == 2);
}
\end{lstlisting}
\end{subfigure}
\end{figure}

Input to both of these functions is similar to that of C's
\lstinline|printf()| function; ``@'' symbols in the
\lstinline|format| string are replaced by arguments from the list
``...''.  Figure~\ref{fig:cinjToBoogie} demonstrates the C level usage,
and Boogie level translations of both of these functions.

My implementation of pthread support for SMACK performs modeling
exclusively at the C code level, using these Boogie injection
functions to model environmental state where necessary.

\section{Thread Creation}
[Good]
One of the critical tasks in implementing support for pthreads in
SMACK was modeling the environment that the operating system provides.
For each thread, the OS tracks the thread ID and thread status of
each thread spawned within a process.  This information is tracked by
the operating system in a thread control block (TCB).  At any time, a
thread must be able to query for its thread ID.  Similarly, a 
parent thread must be able to query the operating system for the
running state threads it has spawned.

This section discusses the challenges of modeling the environment
presented by the operating system to the pthread library, followed by
a description of the model created to implement pthread creation
within SMACK.

\subsection{Modeling the Environment}
[Good]
Faithfully modeling the environment seen by the pthread library
requires addressing a number of bookkeeping issues.  These include
tracking the running state of each thread, and coordinating the thread
ID between the C level code and Corral itself.  The critical issue can
be illustrated by considering the API call
\lstinline|pthread_join(pthread_t* thr)|. This call blocks the calling
thread until the thread specified by \lstinline|thr| is in the stopped
state.  Consequently, properly model this function requires that the
calling thread must be able to query the environment for the status of
the thread referenced by \lstinline|thr|, in order to allow the
calling thread to unblock execution. 

Within the context of static model checking, because no code is
actually executing, the running state of a thread is simply a bit of
program state information - a thread can be marked as waiting,
running, or stopped. Thus, tracking thread states requires the ability
to reference some bit of state information about the threads. A
\lstinline|pthread_t| object passed into \lstinline|pthread_join()|,
for example, must have some method of referencing this bit of thread
state information.  It seems natural to store each thread's thread ID
in this \lstinline|pthread_t| object, and use this to look up the
thread's state in an array.  Indeed, this is precisely how the
operating system tracks such information.  In addition, such an array
must be globally accessible, because threads must be able to query the
state of another thread (to singal \lstinline|pthread_join()|, for
example).  To model this, a global Boogie-level array called
\lstinline|$pthreadStatus| is added to the translated Boogie model of
input pthread programs, as illustrated in
Figure~\ref{fig:pthreadStatusDecl}.

[Figure out better typesetting of these small guys]
\begin{figure}[h]
\centering
\caption{Declaring Global \$pthreadStatus}\label{fig:pthreadStatusDecl}
\begin{subfigure}[b]{1\textwidth}
\centering
\caption{C Declaration of \$pthreadStatus}\label{fig:pthreadStatusDecl_a}
\begin{lstlisting}
__SMACK_top_decl("var $pthreadStatus: [int][int]int;";)
\end{lstlisting}
\end{subfigure}
~
\begin{subfigure}[b]{1\textwidth}
\centering
\caption{Boogie Translation (as Global Declaration)}
\label{fig:pthreadStatusDecl_b}
\begin{lstlisting}[language=boogie]
var $pthreadStatus: [int][int]int;
\end{lstlisting}
\end{subfigure}
\end{figure}

In order to access thread state from \lstinline|$pthreadStatus|, a
queried thread's thread ID is used to index into the array.  This
requires that a thread's ID is referenced by its
\lstinline|pthread_t| object.  This requires immediately querying
Corral for the thread ID of newly spawned thread, and recording this
information in its \lstinline|pthread_t|.  This is done from the
parent with \lstinline|corral_GetChildThreadID()|, and from the child
with \lstinline|corral_GetThreadID()|.  Now, when
\lstinline|pthread_join()|, for example, queries Corral for the
running state of a thread, it can pass in the thread ID to Corral,
which will use it to index into \lstinline|$pthreadStatus|.

However, there is further complication with implementing this
array. In Boogie, undefined variables take on a nondeterministic value
- that is, Corral assumes they can have any value.  Without
initializing \lstinline|$pthreadStatus|, Corral must consider the case
that threads start in a \emph{stopped} state.  This means that a call
to \lstinline|pthread_join()| immediately following a call to
\lstinline|pthread_create()| could query for the status of the newly
spawned thread before the new thread has initially set its status to
waiting, and see it is in the stopped state, causing the parent thread
to unblock.  To address this, \lstinline|$pthreadStatus| is
initialized so that all indices indicate an \emph{uninitialized} state,
as illustrated in Figure~\ref{fig:pthreadStatusInit}.  This code gets
executed as the first statement in \lstinline|main()|, to ensure that
any thread's running state is defined prior to its use.

\begin{figure}[h]
\centering
\caption{Initializing Global \$pthreadStatus}\label{fig:pthreadStatusInit}
\begin{subfigure}[b]{1\textwidth}
\centering
\caption{C Initialization of \$pthreadStatus}\label{fig:pthreadStatusInit_a}
\begin{lstlisting}
__SMACK_code("assume (forall i:int :: "
                             "$pthreadStatus[i][0] == "
                             "$pthread_uninitialized);"); 
\end{lstlisting}
\end{subfigure}
~
\begin{subfigure}[b]{1\textwidth}
\centering
\caption{Boogie Translation}\label{fig:pthreadStatusInit_b}
\begin{lstlisting}[language=boogie]
assume (forall i:int :: $pthreadStatus[i][0] == 
                        $pthread_uninitialized);
\end{lstlisting}
\end{subfigure}
\end{figure}

With these Boogie level additions to each pthread program's Boogie
translation, the operating system environment is sufficiently prepared
to enable the modeling of the actual pthread library behavior.

\subsection{Modeling Thread Creation}
Modeling actual pthread library function call behavior is unlocked
once the OS environment has been modeled.  In the pthread library, the
procedure for spawning threads is \lstinline|pthread_create()|.  There
are two aspects of \lstinline|pthread_create()| that require modeling:
behavior in the calling thread, and behavior in the newly spawned
thread.

In the calling thread, there are two main tasks.  For the first task,
the new thread must be spawned, and instructed to execute the
procedure indicated by the \lstinline|*__start_routine| function
pointer.  For the second, SMACK must query Corral for the thread ID of
the newly spawned child thread, and record this value in the
\lstinline|pthread_t| struct for future reference.  This results in
the model defined in the C code seen in Figure~\ref{fig:pthread_create}.

\begin{figure}[h]
\centering
\caption{Model: pthread\_create()}\label{fig:pthread_create}
\begin{lstlisting}
int pthread_create(pthread_t* __newthread,
                   pthread_attr_t* __attr,
                   void* (*__start_routine) (void *),
                   void* __arg)
{
  __SMACK_code("async call @(@, @, @);", __call_wrapper,
                                         __newthread,
                                         __start_routine,
                                         __arg);

  __SMACK_code("call @ := corral_getChildThreadID();",
               *__newthread);

  return 0;
}
\end{lstlisting}
\end{figure}

In the newly spawned thread, again there is bookkeeping to do, as well
as executing the specified \lstinline|__start_routine()|.  Notice in
Figure~\ref{fig:pthread_create} that the function
\lstinline|__call_wrapper()| is the procedure being executed
asynchronously, rather than \lstinline|__start_routine()|.  This
allows for the child thread to perform bookkeeping before and after
calling \lstinline|__start_routine()|.

As seen in Figure~\ref{fig:__call_wrapper} the newly spawned thread
queries Corral for its thread ID, and then makes a call to
\lstinline|assume()|, which causes the new thread to wait until its
parent has set the child thread's ID in \lstinline|*__newthread|. Once
this has occurred, both the parent and the child have a reference to
the newly spawned thread's ID, and can use this to index into
\lstinline|$pthreadStatus| to set and query the child thread's running
state information. 

Once the child and parent can both reference the correct
\lstinline|$pthreadStatus| element, the new thread can proceed to
executing the specified \lstinline|__start_routine()|.  The new thread
cycles through the \emph{waiting} state to the \emph{running} state,
and then makes a synchronous call to the
\lstinline|__start_routine()|.  After the
\lstinline|__start_routine()| returns, the thread's running state is
finally set to \emph{stopped}, which signals the end of the child
thread. 

\begin{figure}[h]
\centering
\caption{Model: \_\_call\_wrapper()}\label{fig:__call_wrapper}
\begin{lstlisting}
void __call_wrapper(pthread_t* __newthread,
                    void* (*__start_routine)(void *),
                    void* __arg)
{
  int ctid = __SMACK_nondet();
  __SMACK_code("call @ := corral_getThreadID();", ctid);
  assume(ctid == *__newthread);
  
  __SMACK_code("$pthreadStatus[@][0] := $pthread_waiting;",
               *__newthread);
  __SMACK_code("$pthreadStatus[@][0] := $pthread_running;",
               *__newthread);

  __start_routine(__arg);

  __SMACK_code("$pthreadStatus[@][0] := $pthread_stopped;",
               *__newthread);
}
\end{lstlisting}
\end{figure}



\section{Locking Primitives}
These primitives are fairly straightforward, but there are a few
gotchas.  The biggest one is [I am so tired and writing so poorly]
that support for atomic sections is implemented as one single global
lock.   

Of all sections of code enclosed in an atomic begin/end block, only
one can be executing at any given time.  This means using the Corral
atomic primitives alone, there is no way to create multiple mutual
exclusion mechanisms. 

\section{[wc Complex] Synchronization}

\section{Thread Termination}

[Leftovers]
[NEW:Modeling thread creation is/was the most difficult portion of
this project/modeling concurrency.  OLD:There are a number of
challenging aspects of modeling the thread creation process.]  It is
at this stage that the overall model of 
concurrency must be considered and accounted for.  For example,
concurrent calls to a driver may be driven by hardware interrupts and
execute within the kernel, where the process ID is always the same and
unimportant to the execution of the driver.  Alternatively,
multi-threaded user-mode programs often assign different tasks to
threads based on their thread ID, in which case tracking thread IDs
becomes relevant.  The environment within which concurrent threads of
execution run must be modeled prior to thread creation to provide the
framework in which threads will execute [bad sentence, idea is set up
environment so threads can interact with environment as they run] 

[transition]


%%% Local Variables: 
%%% mode: latex
%%% TeX-master: "thesis"
%%% End: 
