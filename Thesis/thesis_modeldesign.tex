\chapter{Model Design}\label{ch:modeldesign}
This chapter discusses research efforts toward developing a model of
the pthread library for SMACK.  It begins by introducing the challenge
of modeling the execution environment presented by the operating 
system to the pthread library.  Having introduced this, it proceeds to
discuss the process of modeling concurrency in the pthread library
over the lifecycle of a pthread. 

\section{Execution Environment}\label{sec:executionenvironment}
One of the critical tasks in implementing support for pthreads in
SMACK was modeling the environment that the operating system provides.
For each thread, the OS tracks the thread ID and thread status of
each thread spawned within a process.  This information is tracked by
the operating system in a thread control block (TCB).  At any time, a
thread must be able to query for its thread ID.  Similarly, a 
parent thread must be able to query the operating system for the
running state threads of it has spawned.

Faithfully modeling the environment seen by the pthread library
requires addressing a number of bookkeeping issues.  These include
tracking the running state of each thread, and coordinating the thread
ID between the C level code and Corral itself.  The critical issue can
be illustrated by considering the API call
\lstinline|pthread_join(pthread_t* thr)|. This call blocks the calling
thread until the thread referenced by \lstinline|thr| is in the stopped
state.  Consequently, properly modeling this function requires that the
calling thread must be able to query the environment for the running
state of the thread referenced by \lstinline|thr|, in order to allow
the calling thread to unblock execution.

Within the context of static model checking, the running state of a
thread is simply a bit of program state information -- a thread can be
marked as waiting, running, or stopped. Thus, tracking thread states
requires the ability to reference some bit of state information about
the threads. Such an array must be globally accessible, because threads
must be able to query the state of another thread (to signal
\lstinline|pthread_join()|, for example).  To track thread running
states in the model, C level code was used to inject a global
Boogie-level array called \lstinline|$pthreadStatus|; the C level call   

\begin{lstlisting}[frame=none,xleftmargin=2\parindent]
__SMACK_top_decl("var $pthreadStatus: [int][int]int;");
\end{lstlisting}

\noindent becomes

\begin{lstlisting}[frame=none,xleftmargin=2\parindent,language=boogie]
var $pthreadStatus: [int][int]int;
\end{lstlisting}

\noindent in the output Boogie program.

In order to access a thread's running state from
\lstinline|$pthreadStatus|, the queried thread's ID is used to index
into the array.  To accomplish this, \lstinline|pthread_t| objects at
the C level must somehow  reference the Corral level thread IDs. It
seems natural to store each thread's ID directly in this
\lstinline|pthread_t| object. Immediately after each thread is
spawned, the model must query Corral for the thread ID of the new
thread, and record its thread ID in the \lstinline|pthread_t|
representing the thread.  This query is performed from the parent by
calling \lstinline|corral_GetChildThreadID()|, and from the child by
calling \lstinline|corral_GetThreadID()|.  With each thread's ID
stored in its corresponding \lstinline|pthread_t| struct, when
\lstinline|pthread_join()|, for example, queries Corral for the
running state of a thread, it can pass in the thread ID as a
parameter, which will be used to index into
\lstinline|$pthreadStatus|. 

However, there is further complication with implementing this
array. In Boogie, undefined variables take on a nondeterministic value
-- that is, Corral assumes they can have any value.  Without
initializing \lstinline|$pthreadStatus|, Corral must consider the case
that threads start in a \emph{stopped} state.  This means that a call
to \lstinline|pthread_join()| immediately following a call to
\lstinline|pthread_create()| could query for the status of the newly
spawned thread before the new thread has initially set its status to
\emph{waiting}.  Such an interleaving of parent and child thread
instructions would force Corral to consider the case where the parent
sees the child as being in the \emph{stopped} state, since
\lstinline|$pthreadStatus| at the child thread's index would be
uninitialized and nondeterministic in value.  This would allow the
parent to unblock before the child ever executed a single
instruction, since under these conditions it would be possible for
\lstinline|$pthreadStatus| to have a value of \emph{stopped} in the
child thread's index. To address this,\lstinline|$pthreadStatus| is
initialized so that all indices indicate an \emph{uninitialized}
state; the C level call  
 
\begin{lstlisting}[frame=none,xleftmargin=2\parindent]
__SMACK_code("assume (forall i:int :: "
                             "$pthreadStatus[i][0] == "
                             "$pthread_uninitialized);"); 
\end{lstlisting}

\noindent becomes 

\begin{lstlisting}[frame=none,xleftmargin=2\parindent,language=boogie]
assume (forall i:int :: $pthreadStatus[i][0] == 
                        $pthread_uninitialized);
\end{lstlisting}

\noindent in the output Boogie program.  This code gets executed as
the first statement in \lstinline|main()|, to ensure that any thread's
running state is defined prior to its use. 

\section{Thread Creation}
With the pthread library's execution environment modeled, a sufficient
framework exists to begin modeling the behavior of the actual pthread
library function calls.  In the pthread library, the procedure for
spawning threads is \lstinline|pthread_create()|.  There are two
aspects of \lstinline|pthread_create()| that require modeling:
behavior in the calling thread, and behavior in the newly spawned
thread. 

In the calling thread, there are two main tasks.  As the first task,
the new thread must be spawned, and instructed to execute the
procedure indicated by the \lstinline|*__start_routine| function
pointer.  As the second, SMACK must query Corral for the thread ID of
the newly spawned child thread, and record this value in the
\lstinline|pthread_t| struct for future reference.  This results in
the model defined in the C code seen in Figure~\ref{fig:pthread_create}.

\begin{figure}[!ht]
\centering
\begin{lstlisting}
int pthread_create(pthread_t* __newthread,
                   pthread_attr_t* __attr,
                   void* (*__start_routine) (void *),
                   void* __arg)
{
  __SMACK_code("async call @(@, @, @);", __call_wrapper,
                                         __newthread,
                                         __start_routine,
                                         __arg);

  __SMACK_code("call @ := corral_getChildThreadID();",
               *__newthread);

  return 0;
}
\end{lstlisting}
\caption{Model: pthread\_create()}\label{fig:pthread_create}
\end{figure}

In the newly spawned thread, again there is bookkeeping to do, as well
as executing the specified \lstinline|__start_routine()|.  Notice in
Figure~\ref{fig:pthread_create} that the function
\lstinline|__call_wrapper()| is the procedure being executed
asynchronously, rather than \lstinline|__start_routine()|.  This
allows for the child thread to perform bookkeeping before and after
calling \lstinline|__start_routine()|.

As seen in Figure~\ref{fig:__call_wrapper} the newly spawned thread
queries Corral for its thread ID, and then makes a call to
\lstinline|assume()|, which causes Corral to only consider
interleavings of the parent and child such that
\lstinline|*__newthread| already contains the thread ID of the child
thread -- effectively, the new thread ``waits'' until its parent has
set the child thread's ID in \lstinline|*__newthread|. Once this has
occurred, both the parent and the child have a reference to the newly
spawned thread's ID, and can use this to index into
\lstinline|$pthreadStatus| to set and query the child thread's running
state information.

\begin{figure}[!ht]
\centering
\begin{lstlisting}
void __call_wrapper(pthread_t* __newthread,
                    void* (*__start_routine)(void *),
                    void* __arg)
{
  int ctid = __SMACK_nondet();
  __SMACK_code("call @ := corral_getThreadID();", ctid);
  assume(ctid == *__newthread);
  
  __SMACK_code("$pthreadStatus[@][0] := $pthread_waiting;",
               *__newthread);
  __SMACK_code("$pthreadStatus[@][0] := $pthread_running;",
               *__newthread);

  __start_routine(__arg);

  __SMACK_code("$pthreadStatus[@][0] := $pthread_stopped;",
               *__newthread);
}
\end{lstlisting}
\caption{Model: \_\_call\_wrapper()}\label{fig:__call_wrapper}
\end{figure}

Once the child and parent can both reference the correct
\lstinline|$pthreadStatus| element, the new thread can proceed to
executing the specified \lstinline|__start_routine()|.  The new thread
cycles through the \emph{waiting} state to the \emph{running} state,
and then makes a synchronous call to the
\lstinline|__start_routine()|.  After the
\lstinline|__start_routine()| returns, the thread's running state is
finally set to \emph{stopped}, which signals the end of the child
thread. 

\section{Thread Execution}
With the ability to spawn threads, the next consideration is the 
functionality of the pthread library that pertains to an executing
thread.  Among this functionality, the most commonly utilized are the
synchronization routines. 

Corral extends Boogie IVL with two primitives for achieving
synchronization between threads: \lstinline|corral_atomic_begin()| and
\lstinline|corral_atomic_end()|.  These calls provide the ability to
create critical sections of Boogie IVL code, such that the
instructions of an atomic block cannot be interleaved with those of
any other atomic block.  However, these atomic blocks are global, and
take no parameters.  There is no way to differentiate atomic blocks of
unrelated code that should be allowed to run concurrently -- this is
much like the big kernel lock in Linux.  This requires the more
complex synchronization routines of pthreads be modeled using
Corral's atomic primitives as building blocks.  

\subsection{Modeling Mutexes}
Mutexes are the most commonly used mechanisms for providing
synchronization among threads.  In addition, they can be used as
building blocks to devise more complex synchronization mechanisms.
Modeling mutexes in Corral is relatively straightforward, and more
complex synchronization functionality easily builds on this
foundation. 

Each \lstinline|pthread_mutex_t| object allocated creates an
independent mutual exclusion device.  To track the current owner of a
mutex and prevent execution of threads that do not hold the lock, the
model stores the thread ID of the current mutex owner in the
\lstinline|pthread_mutex_t| struct.  This allows
\lstinline|pthread_mutex_lock()| to block execution until until the
mutex is in the \emph{UNLOCKED} state.


\begin{figure}[!ht]
\centering
\begin{lstlisting}
int pthread_mutex_lock(pthread_mutex_t* __mutex)
{
  int tid = (int)pthread_self();
  assert(__mutex->init == INITIALIZED);
  assert(__mutex->lock != tid);

  __SMACK_code("call corral_atomic_begin();");
  assume(__mutex->lock == UNLOCKED);
  __mutex->lock = tid;
  __SMACK_code("call corral_atomic_end();");

  return 0;
}
\end{lstlisting}
\caption{Model: pthread\_mutex\_lock()}\label{fig:pthread_mutex_lock}
\end{figure}

This leads to the straightforward model shown in
Figure~\ref{fig:pthread_mutex_lock}. The calling thread queries for
its thread ID using \lstinline|pthread_self()| (which simply queries
Corral with a call to \lstinline|corral_GetThreadID()|, and returns
the result). The mutex is then checked to ensure it has been
initialized, and that the calling thread is not already the owner of
the mutex.  With these criteria ensured, the Boogie translation
atomically waits for the mutex to enter the \emph{UNLOCKED} state and
updates the mutex owner to be the calling thread's ID. 

\begin{figure}[!ht]
\centering
\begin{lstlisting}
int pthread_mutex_unlock(pthread_mutex_t* __mutex)
{
  int tid = (int)pthread_self();
  assert(__mutex->init == INITIALIZED);
  assert(__mutex->lock == tid);

  __SMACK_code("call corral_atomic_begin();");
  __mutex->lock = UNLOCKED;
  __SMACK_code("call corral_atomic_end();");

  return 0;
}
\end{lstlisting}
\caption{Model: pthread\_mutex\_unlock()}
\label{fig:pthread_mutex_unlock}
\end{figure}

The procedure \lstinline|pthread_mutex_unlock()|, seen in
Figure~\ref{fig:pthread_mutex_unlock}, releases a mutex, and is
essentially the inverse of \lstinline|pthread_mutex_lock()|; the model
ensures the mutex is initialized and the calling thread is the current
owner, and atomically sets the mutex as being unlocked. 

\subsection{Complex Synchronization}
The pthread library also includes a number of more complex
synchronization mechanisms.  These allow for more efficient runtime
behavior, in addition to providing commonly utilized synchronization
mechanisms.

The call \lstinline|pthread_cond_wait(pthread_cond_t* cond, |
\lstinline|pthread_mutex_t* mutex)|  is one such function that
provides more efficient runtime behavior.  It blocks the calling
thread until both the \lstinline|cond| predicate is true \emph{and}
\lstinline|mutex| is \emph{UNLOCKED}, at which point the thread is
signaled to resume. This avoids having to continually acquire the
mutex and evaluate the predicate.  The procedure
\lstinline|pthread_barrier_wait()| is a good example of a procedure
providing expected functionality.  It causes calling threads to block
until all threads calling \lstinline|pthread_barrier_wait()| have
reached this instruction. These more complex synchronization routines
are modeled entirely as C level code simply by using
\lstinline|pthread_mutex_t|'s as building blocks. 

\section{Thread Termination}
As a thread's execution comes to an end, there is a bit of bookkeeping
to be done to wrap up the thread's lifecycle.  It is primarily this
termination handling that required the upfront work of modeling the
operating system environment prior to implementing support for thread
creation.

There are two main considerations for handling the termination of a
thread.  The first consideration is the passing of a return value from
a terminating thread.  The second is blocking execution of a waiting
thread, and retrieving the return value of the terminating thread
being waited on.

\subsection{Modeling Thread Exit and Join}
It is often necessary to pass a return value from a terminating thread
to some other thread.  For example, with a scatter/gather design
pattern, a parent thread will send data to multiple worker threads,
which return a computational result back to the parent.

The challenge in achieving this was that this return data has to
exist somewhere between the time that the terminating thread exits
and the joining thread retrieves it.  This functionality was
implemented by extending the definition of the
\lstinline|$pthreadStatus| array to store a tuple rather than a 
single value.  With this, the model can store the return value passed
into \lstinline|pthread_exit(void* value)| as the second tuple 
element, in addition to storing the thread's running state as the
first element.

\begin{figure}[!ht]
\centering
\begin{lstlisting}
void pthread_exit(void *retval)
{
  pthread_t tid = __SMACK_nondet();
  tid = pthread_self();

  __SMACK_code("assert $pthreadStatus[@][0] == "
               "$pthread_running;",
               tid);

  __SMACK_code("$pthreadStatus[@][1] := @;", tid, retval);

  __SMACK_code("$pthreadStatus[@][0] := $pthread_stopped;",
               tid);
}
\end{lstlisting}
\caption{Model: pthread\_exit()}\label{fig:pthread_exit}
\end{figure}

The model of \lstinline|pthread_exit()|, seen in
Figure~\ref{fig:pthread_exit}, gets the calling thread's ID and
ensures the thread is still in a running state, saves the return value
pointer in \lstinline|$pthreadStatus|, and finally sets the running
state to \emph{stopped}.

With the return value from threads now stored in
\lstinline|$pthreadStatus|, a method for accessing this value is
needed.  This is done through 
\lstinline[breaklines]|pthread_join(pthread_t* thr, void** __thread_return)|,
which blocks the calling thread until the thread referenced by 
\lstinline|thr| terminates. The pointer \lstinline|__thread_return| is
set to the pointer passed in to \lstinline|pthread_exit()| by the
terminating thread (or \lstinline|null| if no such call was made by
the terminating thread), providing the thread calling
\lstinline|pthread_join| with a reference to the terminating thread's
return value.

\begin{figure}[!ht]
\centering
\begin{lstlisting}
int pthread_join(pthread_t __th, void **__thread_return)
{
  __SMACK_code("assume $pthreadStatus[@][0] == "
               "$pthread_stopped;",
               __th);

  __SMACK_code("@ := $pthreadStatus[@][1];",
               *__thread_return, __th);

  return 0;
}
\end{lstlisting}
\caption{Model: pthread\_join()}\label{fig:pthread_join}
\end{figure}

The model designed for \lstinline|pthread_join()|, shown in
Figure~\ref{fig:pthread_join} implements this behavior.  A call to
\lstinline|assume()| causes the thread to block until the target
thread's running state is \emph{stopped}.  Once the terminating thread
is stopped, the waiting thread fetches the return value pointer from
the \lstinline|$pthreadStatus| array, which is then assigned to the
pointer \lstinline|__thread_return|.



%%% Local Variables: 
%%% mode: latex
%%% TeX-master: "thesis"
%%% End: 
