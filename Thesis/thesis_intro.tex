\chapter{Introduction}\label{thesis_intro}

Verification of programs using formal methods has long been a promising area whose practical adoption has gone unrealized.  Recent advances in computing performance and verification algorithms have made these well-understood formal verification techniques available for real world applications.  As practical adoption of formal verification becomes a reality, it has become necessary to create tools that [wc address] the use cases where formal verification will be most advantageous.

One such use case is the verification of concurrent programs.  As multi-core systems become more ubiquitous, the parallel programming paradigm is increasingly relevant.  Parallel programming presents a unique challenge for developers, as an extra dimension of complexity is introduced.  Indeed, concurrent programs can suffer from a host of issues that simply do not apply to the sequential programming paradigm, such as race conditions and deadlocks.  As a result, verification tools suited for real world usage should support concurrency.

SMACK is a bounded software verification tool for C/C++ programs, and is the result of a joint effort between the University of Utah, IMDEA Software Institute, and Microsoft Research~\cite{smack}.  With a growing user base and active support from both the academic and industry communities, it is a good candidate for continued development.  SMACK, however, currently lacks support for any form of concurrency.  [justify selection of pthreads?]  As such, though pthreads support is implemented in other formal verification tools, the community benefits from extending SMACK with a model to support abstract interpretation of C/C++ programs that utilize the pthreads library.

SMACK utilizes an intermediate verification language (IVL) called Boogie, which separates the semantic modeling of input source programs from the processes and algorithms involved in verifying modeled input programs.  Naturally then, extending SMACK to support additional [wc input source libraries] requires designing models that express the behavior of the [wc source library calls] using the Boogie IVL.  Designing such models for the pthread API requires investigating and understanding the underlying constructs needed for expressing the behavior of concurrent programming paradigms in general.  There is little published work discussing the design of the underlying building blocks used for modeling concurrency.

[This paragraph = improved disaster]
In this paper, I demonstrate that the complex behavior of the pthread API can be accurately modeled using the very small set of concurrency modeling primitives available in Boogie.  First, I generically discuss the process of modeling common concurrency constructs, using my pthread extension to SMACK as an example. Next, I [describe] the actual implementation of support for pthreads in the SMACK toolchain, allowing for the verification of concurrent programs.  Finally, I highlight the empirical testing performed on the pthread extension to SMACK which demonstrates the accuracy of the implementation.  


[Remaining is leftover stuff - may be useful when revising intro]
This is supported by the accuracy seen in the empirical results of initial implementation testing.  My proposed research will extend SMACK with a model of a common subset of the pthreads API, enabling verification of C/C++ programs utilizing the Pthreads library.  Extending SMACK to provide support for programs using pthreads has not only provided the SMACK community with support for a common concurrency paradigm, but also presents an opportunity to investigate general constructs useful for modeling concurrency.  These constructs, which will necessarily result from extending SMACK, will lend themselves to modeling other concurrency libraries like OpenMP and MPI.  Further, the resulting constructs should be general enough for use within other verification tools.
%%% Local Variables: 
%%% mode: latex
%%% TeX-master: "thesis"
%%% End: 
