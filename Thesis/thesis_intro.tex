\chapter{Introduction}\label{thesis_intro}

Verification of programs using formal methods has long been a
promising area whose practical adoption has gone unrealized.  Recent
advances in computing performance and verification algorithms have
made these well-understood formal verification techniques available
for real world applications.  As practical adoption of formal
verification becomes a reality, it has become necessary to create
tools that address the use cases where formal verification will
be most advantageous. 

One such use case is the verification of concurrent programs.  As
multi-core systems become more ubiquitous, the parallel programming
paradigm is increasingly relevant.  Parallel programming presents a
unique challenge for developers, as an extra dimension of complexity
is introduced.  Indeed, concurrent programs can suffer from a host of
issues that simply do not apply to the sequential programming
paradigm, such as race conditions and deadlocks.  As a result,
verification tools suited for real world usage should support
concurrency.

SMACK is a bounded software verification tool for C/C++ programs, and
is the result of a joint effort between the University of Utah, IMDEA
Software Institute, and Microsoft Research~\cite{smack}.  With a
growing user base and active support from both the academic and
industry communities, it is a good candidate for continued
development.  SMACK, however, currently lacks support for any form of
concurrency.  Requests from the SMACK user community have encouraged
the SMACK development team to add support for the pthread library as
its first supported concurrency paradigm.   As such, though pthreads
support is implemented in other formal verification tools, the
community benefits from extending SMACK with a model to support
abstract interpretation of C/C++ programs that utilize the pthreads
library.  

SMACK utilizes an intermediate verification language (IVL) called
Boogie, which separates the semantic modeling of input source programs
from the processes and algorithms involved in verifying modeled input
programs.  Rather than directly model checking libraries included by
input source programs, modeling the behavior of included libraries is
the preferred method for verifying input programs utilizing such
libraries.  This reduces both the computational and engineering
complexity required to verify input programs including such libraries,
as details such as the underlying hardware, operating system and error
handling are abstracted away in the models being checked.

It follows, then, that extending SMACK to support the verification of
input programs that include the pthread library requires designing a
model that express the behavior of the pthread library using the
modeling environment available in SMACK.  Designing such a model for
the pthread API requires investigating and understanding the
underlying constructs needed for expressing the behavior of concurrent
programming paradigms in general.  This leads to the thesis of this
paper:

\begin{quote}
\begin{center}
\textbf{Thesis Statement}
\end{center}
``The complex behavior of the pthread
API can be accurately modeled using the set of basic concurrency
modeling primitives available in the SMACK modeling environment,
allowing SMACK to accurately verify input programs utilizing the
pthread library.''
\end{quote}

In this paper, I demonstrate that SMACK can indeed be used to
accurately model the pthread library.  First, I discuss in further
detail the SMACK verification framework.  Having introduced SMACK, I
proceed to detail the process of developing a model which captures the
behavior of the pthread library function calls. Next, I describe the
actual implementation of support for pthreads in the SMACK toolchain,
allowing for the verification of concurrent programs.  Finally, I
highlight the empirical testing performed on the pthread extension to
SMACK which demonstrates the accuracy of the implementation. 

%%% Local Variables: 
%%% mode: latex
%%% TeX-master: "thesis"
%%% End: 
