\chapter{Background}{Test}
[Discuss motivation for boogie, IVL in general]

SMACK itself is essentially a compiler that takes C/C++ programs and translates them into BoogiePL (or simply Boogie), an intermediate verification language (IVL)~\cite{smack}.  This converted Boogie code is then consumed by a static analysis tool that evaluates verification conditions present in the original source code.

The SMACK toolset is a front-end for a program verification toolchain that features the Corral program verifier as its core back-end.  Input C/C++ programs are given to Clang to compile and link, resulting in an LLVM bytecode output.  This is then passed to SMACK which translates the LLVM bytecode into a Boogie program that models the behavior of the input C/C++ program.  The resulting Boogie program is then passed to Corral.  Corral converts the Boogie program into an SMT query, which is given to the Z3 SMT solver for evaluation.

Boogie is a low level modeling language.  It contains support for little more than a typing system, basic arithmetic and boolean expressions \& statements, control flow \& procedure calls, and verification condition specification~\cite{boogie}.  Any more complicated semantics of the source language must be modeled using the basic set of primitives available in the language.  Though this gives freedom to support a large variety of models of computation and source languages, it requires that models be created to define the semantics of operations available within the source languages and computational models.

As an example, there is no concept of a heap within Boogie.  Because of this, a memory model must be developed that accurately models the behavior of the heap.  A rudimentary model of memory could use a simple, large array of integers to represent the heap, where each element represents a word of memory, and C pointers are simply indices into this array.

[Does this belong somewhere else?] It should be noted that there exists a back-end verifier from Microsoft Research named Boogie.  The Boogie program verifier provides similar functionality to the Corral program verifier.  Hereafter, ``Boogie'' will refer to BoogiePL unless otherwise specified.

The Corral program verifier provides an extension to the Boogie language that includes a very basic set of primitives for handling concurrency.  This extension includes the following calls:
\begin{itemize}
\item \lstinline|async call| \emph{func}\lstinline|(|\emph{...}\lstinline|)| - Asynchronously calls \emph{func} with the parameter list \emph{'...'}
\item \lstinline|corral_atomic_begin()| - Begins an atomic block
\item \lstinline|corral_atomic_end()| - Ends an atomic block
\item \lstinline|corral_getThreadID()| - Returns the ID of the calling thread.
\item \lstinline|corral_getChildThreadID()| - Returns the ID of the thread most recently spawned in the calling procedure.
\end{itemize}
The Pthreads API is much more complex than these concurrency primitives recognized by Corral.  As a result, to provide support for the more complex Pthreads API within SMACK, it is necessary to model the behavior of the Pthreads API using the primitives provided by Corral.

There are several projects, including Inspect and CIVL, that provide support for Pthreads~\cite{civl}\cite{inspect}.  I will be referring to their implementations, as well as using their benchmarks and regression tests to assess the accuracy of the new implementation within SMACK.
%%% Local Variables: 
%%% mode: latex
%%% TeX-master: "thesis"
%%% End: 
